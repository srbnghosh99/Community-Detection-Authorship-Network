\documentclass{article}
\usepackage{graphicx} % Required for inserting images

\title{Sciencemetrics}
\author{Shrabani Ghosh}
\date{April 2023}

\begin{document}

\maketitle

\section{Content}

\subsection{ 2022  Most papers have used GNN for graph-related problems like node classification, link prediction and clustering, recommendation systems, and few-shot learning.}
\subsection{2021 Most of the paper has done community detection}
\subsection{Authorship Network Community Detection}

The collaboration culture has changed over years in computer science research fields. Each research field community might have a specific pattern of community formation. We can see a couple of approaches to model co-authorship information for community detection. 
\begin{enumerate}
    \item The structure of different research communities and how communities changed over the past 10 years. We can uncover highly connected and lowly connected users from co-authorship networks. Highly connected nodes can be considered influential authors. And we can explore how it would be interesting to see the evaluation of communities. We can find some communities are growing and some are merging with other research field communities. \cite{mandaglio2019dynamic}  
    \item In the research field, researchers collaborate in interdisciplinary research areas from all over the world. We can explore overlapping community structures. Some nodes might be part of multiple communities. Also removing a node might variate the strength of the structure. This paper \cite{rajeh2021identifying}has done "Overlapping Modularity Vitality" to identify a node based on the node's contribution to a network. 
\end{enumerate}

\subsection{Gender and Racial Bias in Research Communities}
This paper \cite{mckay2022and} examined the participation effects at HCI conferences based on two axes of diversity: gender and geographic location. This work has broadly discussed the participation of men, women, non-binary and transgender people in academic conferences. Also, the discussion covered how the geographical location of conferences and the pandemic situation have impacted of women's participation in three major CHI conferences. The globe's geographical regions are divided into 8 regions: North America (USA, Canada, Mexico); Central and South America; Nordic Europe; non-Nordic Europe; the Middle East; Asia and Australasia. The geographical information is collected from authors' affiliations.  
Similarly, we can explore gender and racial differences in authors of different research areas in computer science. \cite{molwitz2023female}

\subsection{Survey Paper Publication Venue}
\begin{enumerate}
    \item IEEE TRANSACTIONS ON BIG DATA \cite{wang2022survey}
    \item ACM Computing Surveys \cite{tang2016survey}
    \item IEEE TRANSACTIONS ON KNOWLEDGE AND DATA ENGINEERING \cite{shi2016survey}
    \item IEEE TRANSACTIONS ON NEURAL NETWORKS AND LEARNING SYSTEMS \cite{su2022comprehensive}
    
\end{enumerate}
\subsection{Survey Paper on Graph Visualization }
\begin{enumerate}
    \item Community extraction and visualization in social networks applied to Twitter \cite{abdelsadek2018community}
    \item Text Visualization Techniques: Taxonomy, Visual Survey, and Community Insights  \cite{kucher2015text}
    \item Clustering and community detection in directed networks: A survey \cite{malliaros2013clustering}
    \item What Would a Graph Look Like in This Layout? A Machine Learning Approach to Large Graph Visualization \cite{kwon2017would}
 \end{enumerate}  

\subsection{Survey Paper on Community Detection}
\begin{enumerate}
    \item A Survey of Community Detection Approaches: From Statistical Modeling to Deep Learning \cite{jin2021survey}
    \item A Comprehensive Survey on Community Detection WIth Deep Learning \cite{su2022comprehensive}
    \item Clustering and community detection in directed networks: A survey \cite{malliaros2013clustering}
    \item Community Detection Related Survey \url{https://github.com/RapidsAtHKUST/CommunityDetectionCodes/tree/master/Survey}
 \end{enumerate}  

\subsection{Github Repositories of Community Detection}
\begin{enumerate}
    \item \url{https://github.com/topics/community-detection-algorithms}
    \item \url{https://github.com/shobrook/communities}
    \item \url{https://github.com/topics/louvain-algorithm}
    \item Non-OverlappingCodes \url{https://github.com/RapidsAtHKUST/CommunityDetectionCodes/tree/master/NonOverlappingCodes}
 \end{enumerate}  



%  \section{\textbf{ \LARGE Dataset}}

 \subsection{DBLP}
 DBLP database have four options to view articles. i) persons
 ii) conferences iii) journals iv) series

 i) persons option has the list of all authors in DBLP database
 ii) conferences has the list of conferences of computer science which includes national, international conferences, workshops.
 iii) journal option has the list of journals 
 iv) series includes workshop papers, lecture notes etc. 

 We have collected data based on conferences and journals option. For each computer science research area, we have conferences and journals related to that specific area. For example, for Human Computer Interaction research area, conferences available in DBLP are 
\bibliographystyle{plain}
\bibliography{ref.bib}
\end{document}
